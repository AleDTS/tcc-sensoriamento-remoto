\chapter{Introdução}\label{introducao}

Dados gerados a partir da Observação da Terra são ricas fontes para
descobrir como a Terra está mudando. Imagens obtidas a partir de
satélites que orbitam o globo possibilitam uma visão de conjunto
multitemporal da superfície terrestre, o que possibilita o estudo e
monitoramento dos impactos causados por fenômenos naturais e antrópicos \cite{florenzano2002imagens}. Portanto, o desenvolvimento de técnicas e abordagens que viabilizam a análise desses dados é essencial.

As mudanças na ocupação do solo afetam o clima global, logo torna-se
relevante o devido monitoramento do uso e cobertura da terra em escala
global \cite{wulder2014satellites}. Isso é possível através da \textbf{coleta}, \textbf{distribuição} e \textbf{análise} de dados obtidos por Sensoriamento Remoto.

\begin{citacao}
Sensoriamento remoto é uma técnica de obtenção de imagens dos objetos
da superfície terrestre sem que haja um contato físico de qualquer
espécie entre o sensor e o objeto.\cite[p. 3]{meneses2012introduccao}.
\end{citacao}

\textbf{Cobertura da terra} pode ser definido como a cobertura biofísica observada na superfície terrestre. Já o termo \textbf{uso da terra} abrange os arranjos, atividades e insumos empreendidos pelas pessoas em um tipo de cobertura da terra para produzir, alterar ou manter. \cite{di2016land} 
 
Pesquisas na área de Sensoriamento Remoto envolvendo métodos digitais de classificação de imagens chamam a atenção porque seus resultados são a base para
muitas aplicações ambientais e socioeconômicas \cite{lu-weng}. Além
disso, metodologias de aprendizado de máquina têm bons resultados em
aplicações reais, especialmente para tarefas de classificação e
regressão, uma vez que não é necessário um conhecimento a priori sobre o modelo de distribuição dos dados disponíveis nem o relacionamento entre as variáveis independentes precisam ser assumidos. Essas são
propriedades desejáveis para o sucesso desses métodos para a análise de
imagens de sensoriamento remoto. \cite{waske2009machine}

Hoje em dia, há uma alta demanda para aplicações de alta performance
em geoprocessamento, num contexto onde muitos dados georreferenciados
são produzidos a todo instante, a exemplo dos \emph{smartphones} que
possuem GPS e são amplamente utilizados. Além disso, \sigla{SR}{\emph{Sensoriamento Remoto}} para
monitoramento é uma tarefa que exige um muito processamento e espaço em
disco disponível \cite{geocompr}.

A solução atualmente é a computação e armazenamento em aplicações
baseadas em nuvem, ou seja, serviços disponibilizados por estruturas
capazes de lidar com uma grande quantidade de dados de maneira
eficiente. Com isso, a coleta e distribuição de dados gerados por
sensoriamento remoto se torna muito mais viável, possibilitando o
monitoramento em escala global. \cite{eoosi}

\section{Objetivos}\label{objetivos}

\subsection{Objetivos Gerais}\label{objetivos-gerais}

Levando em consideração a importância da aplicação de métodos
analíticos para monitoramento da cobertura terrestre, bem como os meios
que tornam essa tarefa viável, este trabalho teve como objetivo a
aplicação e estudo de um processo de construção e comparação modelos
preditivos, a fim de realizar a classificação de novos dados.

O desenvolvimento foi realizado a partir da coleta de dados da
iniciativa MapBiomas, e também imagens do sensor MSI do satélite
Sentinel-2, ambos disponíveis gratuitamente. Os ambientes
utilizados foram a IDE RStudio da linguagem R, bem como
plataforma do Google Earth Engine. A área de estudo foi a região
do município de Bauru - SP. Os métodos de classificação de imagens
aplicados foram: Florestas Aleatórias (ou \emph{Random Forests}) e
Máquina de Vetores Suporte (ou \emph{Support Vector Machines}) linear com regularização e também com função de núcleo não linear.

\subsection{Objetivos Específicos}\label{objetivos-especificos}

\begin{itemize}
\itemsep1pt\parskip0pt\parsep0pt
\item
  Revisão teórica: organizar um estudo dos conceitos chave acerca dos
  temas que envolvem o trabalho, ou seja, sensoriamento remoto,
  aprendizado de máquina, classificação de imagens.
\item
  Análise e comparação: discussão dos métodos e dados a serem coletados,
  bem como apresentação das ferramentas utilizadas.
\item
  Análise das classes de uso e cobertura da terra para o bioma Cerrado, a partir da metodologia adotada pela iniciativa MapBiomas
\item
  Implementação: desenvolvimento do modelo e aplicação.
\item
  Teste e validação dos resultados: realizar uma avaliação de precisão
  de classificação, a fim de comparar os modelos aplicados e discussão
  de resultados
\end{itemize}
